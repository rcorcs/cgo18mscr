\vspace{-2em}
\section{Introduction}

Modern optimising compilers have reached a high level of sophistication,
providing a large number of optimisations, where the correct choice of
optimisations and their ordering can have a significant impact on the
performance of the code being optimised.
Although compilers offer a set of prearranged optimisation sequences that are
expected to yield reasonable improvements in many programs, there is still the
potential for performance degradation for certain programs, as these
optimisation sequences do not include all possible optimisations and are always
applied in the same pre-defined order, without regard the code being
optimised~\cite{pan06,cavazos07,zhou12,kulkarni12}.
A well-known compilation technique that addresses this challenge is {\itercomp},
which has the ability to adapt to new platforms, program and workload while
still having a systematic and simple optimisation process~\cite{kisuki99,fursin07,chen10}.
%It works by repeatedly evaluating a large number of optimisation sequences until
%the best combination is found for a particular
%program~\cite{kisuki99,fursin07,chen10}.

%{\Itercomp} is a well-known compilation technique that addresses the problem of
%efficiently selecting the best optimisation sequence for a given program.
%Although compilers offer a set of prearranged optimisation sequences that are
%expected to yield reasonable improvements in many programs, there is still the
%potential for performance degradation for certain programs, as these
%optimisation sequences do not include all possible optimisations and are always
%applied in the same pre-defined order, without regard the code being
%optimised~\cite{pan06,cavazos07,zhou12,kulkarni12}.
%{\Itercomp} have a systematic and simple optimisation process, where it works
%by repeatedly evaluating a large number of optimisation sequences until the best
%combination is found for a particular program~\cite{kisuki99,fursin07,chen10}.

However, until recently, most of the existing work in {\itercomp} had been focusing on
finding the best optimisation through repeated runs using a single input.
Although they demonstrate the potential of {\itercomp}, in real-world
\textit{online} scenarios the user rarely executes a program with the same input
multiple times~\cite{bodin98,kisuki99,stephenson03,kulkarni04,agakov06}.
Furthermore, most of real-world programs are complex enough so that a single
input case does not capture the whole range of possible scenarios and program
behaviours~\cite{haneda06,fursin07,chen10,chen12a}.
Because programs can exhibit behaviours that differ greatly depending on the
input, using a single input for {\itercomp} may not result in a good performance
when executing the optimised code with different inputs.

The main goal of this paper is to enable {\itercomp} in online scenarios.
We define the online scenario as having the restriction that programs execute
multiple inputs and distinct inputs are executed only once.
This online aspect is usually found in mobile and data centre
platforms~\citep{chen12b,fang15,mpeis16}, where the goal is to optimise programs
to consume less resources based on the workload of a particular user or group of
users.

Because of the restriction of having a single execution per input, it is not
possible to measure speedup for comparing optimisations.
Moreover, measuring just execution time, for example, is also not viable since
different inputs often mean different amounts of work,
rendering it meaningless to compare optimisations only by execution time.
%amount of work is often expected to differ between 
%useful only if the
%amount of work is constant between executions with different inputs.
Similarly, although previous works have suggested using
\textit{instructions per cycle} (IPC) for performing {\itercomp} in online
scenarios, IPC also have no correlation with speedup~\citep{fursin07}.

In order to tackle this problem, we propose the a work-efficiency metric that
enables to compare the performance of different optimisations using executions
of the program with distinct inputs.
It works by instrumenting the program, in optimally selected places, using a
single global counter which measures the amount of work the program performed
during its execution.
Equiped with this work metric, we are able to compute the work-efficiency
performance using a single execution of an optimised version of the program,
which can then be used to guide online {\itercomp}.

We acknowledge that having a low-overhead profiling for the work-efficiency
metric is essential in this online scenario for two main reasons:
$(i)$ the user is directly affected by large overheads;
$(ii)$ a highly intrusive instrumentation can have significant impacts on the
effect of optimisations, due to complex and unpredictable decisions taken by
many of the compiler's heuristics.
With the purpose of reducing the profiling's overhead, we propose two relaxation
algorithms which provide a trade-off between measurement accuracy and overhead.
%The first is a relaxation algorithm that operates on the level of regions of
%functions, while the second performs the relaxation considering the whole
%program at the same time.
These relaxation algorithms are able to reduce the overhead by incurring very
small and bounded percenage errors in the measurement of work.
The benefits of reducing the overhead often outweighs the inaccuracy introduced,
because of the two aforementioned drawbacks of large overheads.

Our evaluation shows that performing online {\itercomp} guided by work efficiency
achieves about 80\% of the performance of an offline oracle, with improvements
of up to about 20\% over the compiler's standard optimisation.
While our solution executes each input a single time to measure the work
efficiency metric, the offline oracle is allowed to execute multiple times to
have a statistically sound measurement of the actual speedup.
Regarding the two relaxation algorithms, our results show an average reduction
in overhead of 43\% and $2.1\times$ over the optimal work profiling, observing
up to about $5\times$ of overhead reduction, while often incurring in much less
than 1\% of dynamic error in the measurement of work.

%Our experimental evaluation shows that performing online {\itercomp} guided by the work-based performance (WP) metric good results compared to the oracle, which is allowed to execute each input multiple times in order to use the actual speedup for guiding the {\itercomp}.
%Online {\itercomp} guided by the WP metric is able to achieve an average of 7.5\% and a maximum of 33\% improvement over the standard {\flagstype -O3} optimisation.
%Moreover, the experiments regarding the work profiling show that both relaxation algorithms are able to significantly reduce the profiling overhead while incurring a dynamic error of less than 5\% in the work measurement.
%The whole program relaxation achieves an average of $2\times$ reduction in the overhead compared with the optimal profiling technique, while the more conservative relaxation that operates per region achieves an average improvement of 40\% over the optimal profiling.

%Our main contributions are the following:
%\begin{itemize}
%\item The use of a work-based performance metric in order to enable \textit{online} {\itercomp} by comparing different combination of compiler optimisations even when executed with distinct inputs.
%\item We propose a relaxed instrumentation for low overhead profiling, with a controlled trade-off between accuracy and overhead.
%\end{itemize}

To summarise, the main contributions of this paper are the following:
\begin{itemize}
\item We propose a work efficiency profiling that measures a performance rank
of an optimised program using a single execution.
\item We show the effectiveness of the work efficiency for guiding {\itercomp} in an
online scenario, where the program is expected to execute only once for distinct
inputs.
%\item Contrary to what previous work has suggested, we show that instructions per cycle (IPC) is not a good metric for online {\itercomp}.
%\item We adapt the block frequency profiling in order to measure the WP metric.
\item A conservative relaxation algorithm is proposed for reducing the overhead of the work profiling, with a guaranteed bound for the error.
\item A more aggressive relaxation algorithm is proposed, which works on the whole program, in order to further reduce the overhead of the work profiling.
\end{itemize}

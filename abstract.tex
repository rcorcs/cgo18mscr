\begin{abstract}

    Iterative compilation can dramatically improve the performance of a program by searching for the best optimization.
    However, it relies upon repeatedly evaluating optimizations on predefined representative inputs.
    If the developers fail to guess what is a representative input or usage patterns change over time,
    then the wrong optimizations may be selected.
    Searching the optimization space with actual user inputs, during deployment, has previously seemed infeasible.
    Program side effects and difficulties capturing the system state mean that any given input can
    only be executed once. %Since each input may entail a different amount of work, the search cannot use runtime to compare
    %optimizations, and there is no baseline against which to define speedup.
    Unless we find a way to compare optimizations even when inputs are executed only once, we will be unable to use real user
    input for iterative compilation, users will not have the best optimized programs, and performance will suffer.

    We present metrics and techniques which, for the first time,  enable the use of true {\em online iterative compilation}.
    We compare optimizations by a new work efficiency metric, which requires only a single execution with any one input.
    We develop an instrumented, low overhead, mechanism for gathering the required data, using optimal probe placement in the source
    code. We show that overhead can be further reduced by removing high-cost probes whose absence introduces only small errors to the
    work metric, without significantly affecting the outcome of the iterative compilation.
    We give two variants for probe removal optimization: the first offers strong
    guarantees about the maximum possible error, while the second targets the average whole program error.

    Our online iterative compilation method yields programs that get 80\% of the speedup achievable by an offline oracle,
    but without executing any input more than once, which the oracle cannot do.
    Through probe removal we bring the instrumentation overhead down to only 5\% on average, making our
    approach suitable for real-world scenarios. %We show practical online iterative compilation for the first time,
    %optimizing programs according to the range of inputs users actually present.
\end{abstract}

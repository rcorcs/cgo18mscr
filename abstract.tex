\begin{abstract}
    
    Iterative or adaptive compilation can dramatically improve the performance of a program by searching for the best optimisation
    settings. However, it relies upon repeatedly evaluating and comparing optimisations off-line on representative inputs. If the
    developers fail to guess what is a representative input or usage patterns change over time, then the wrong optimisations may be
    selected. Until this paper, it has been impossible to search online over actual user inputs. Program side effects, interactivity and
    difficulties capturing the system state prevent repeated execution. Since each input may be executed only once and each input may
    entail a different amount of work, the search cannot use runtime to compare optimisations. Unless we find a way to compare
    optimisations even when inputs are executed only once, we will be unable to use real user input for online iterative compilation, users
    will not have the best optimised programs, and performance and energy efficiency will suffer.

    This paper demonstrates online iterative compilation. We compare optimisations by work efficiency, or work divided by time, executing
    each input only once. We develop an instrumented, low overhead, work efficiency metric with optimal probe placement. Then we further
    reduce the overhead by removing high-cost probes whose absence introduces only small errors to the work metric, without significantly
    affecting the outcome of the iterative compilation. We give two variants for probe removal optimisation: the first offers strong
    guarantees about the maximum possible error, while the second targets the average, whole program error.
    
    Our online method yields programs that get 80\% of the speedup an offline oracle achieves but without executing any input more than
    once, which the oracle cannot do. Through probe removal we bring the instrumentation overhead down to only 4\% on average, making our
    approach suitable for regular use. We show practical online iterative compilation for the first time, optimising programs according to
    the range of inputs users actually present.
    
\end{abstract}
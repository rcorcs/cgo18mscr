\begin{abstract}
Until recently, most of the existing work in {\itercomp} had been focusing on
finding the best optimisation through repeated runs using a single input.
However, in real-world online scenarios, the user rarely executes a program with
the same input multiple times and most of the programs are complex enough that a
single input case does not capture the whole range of possible scenarios and
program behaviours.

In this thesis, we focus on enabling {\itercomp} in \textit{online} scenarios.
Because of the restriction on repeating the execution with same input, it is
not always feasible to measure the real speedup when comparing optimisations.
To this end, we propose a \textit{work efficiency} metric to guide
online {\itercomp}, which is able to achieve about 80\% of the performance
of an offline oracle, with improvements of up to 20\%.

Moreover, because we need to instrument the program for measuring work
efficiency, providing a low-overhead profiling is essential.
In order to reduce the profiling overhead, we also propose two relaxation
algorithms which provide a trade-off between measurement accuracy and overhead.
Our results show that the two relaxation algorithms can reduce the average
overhead by 43\% and $2.1\times$ over the optimal work profiling, observing
up to about $5\times$ of overhead reduction.
\end{abstract}

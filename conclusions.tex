\section{Conclusions}
This paper has introduced a novel efficiency metric which enables us to compare in terms of performance versions of an application compiled
in different ways even when they perform different amounts of work. Our approach automatically estimates the work done by each basic block
of the application and inserts instrumentation code to increment a global work counter every time a block is executed. We optimize the
placement of the instrumentation probes to reduce the runtime overhead. On top of it, we implement two strategies for removing probes from
the application to control the trade-off between runtime overhead and work estimation accuracy.

We use this efficiency metric to enable for the first time true online iterative compilation. Instead of comparing different optimizations
sequences offline by running the application repeatedly with different optimizations applied but the same input, we can now compare runs
using different inputs as they are provided by the user. We evaluated our online iterative compilation methodology using the KDataSets
benchmark suite. Experimental results show that we deliver \red{80\%} of the speedup achieved by an offline oracle while not having to
execute any input more than once. Our probe optimization and removal strategies bring the instrumentation overhead down to only \red{4\%}
on average, making our approach suitable for integration into software products.


%This paper has presented a novel work efficiency metric to enable one to compare the performance of any two different compiled versions of
%a program without executing a given input more than once. This ability allows us to implement, for the first time, a true online iterative
%compilation  to search for the best compilation options  based on actual user inputs seen in the deployment environment.

%We develop a compiler-based, low-overhead instrumentation mechanism to gather the require data to evaluate the work metric. Our mechanism
%does this without any human involvement. We show that by carefully selecting where to place the probes in the source code, the
%instrumentation overhead can be negligible. We give two probe optimization strategies, allowing developers to balance the instrumentation
%overhead and the performance error.


%We  evaluate our approach using the KDataSets benchmark suit. Experimental results show that our approach delivers \red{80\%} of the
%speedup achieved by an offline  oracle, but without executing any input more than once. We show that our probe optimization strategy is
%highly effective, which brings the instrumentation overhead down to only \red{4\%} on average, making our approach practical for regular
%use.

%Our proposed technique for relaxed instrumentation can also be applied for similar profiling scenarios, such as profiling empirical
%computational complexity~\cite{goldsmith07,zaparanuks12,coppa14}.

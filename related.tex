\section{Related Work}\label{sec:relatedwork}

\subsection{{\IterComp}}
Various adaptive compilation techniques have been proposed for finding near optimal compilation 
settings~\cite{agakov06,kulkarni04,stephenson03,hoste08}. These approaches compare the performance of candidate compiler options by first
compiling the program with different options and then profiling the different compiled versions on the same program input. All these
approaches implicitly assume that the best-performing compiler settings found from a small set of inputs will work well on unseen datasets.

Chen~\etal~\cite{chen10,chen12a} show that the best optimization options, however, can differ across inputs. Their results suggest that, in
most cases it is possible to find a single set of optimizations that produces good results across all inputs; but achieving this requires
developers provide a few tens of representative inputs to be used by the iterative compilation search process. We argue that providing
representative inputs is difficult in practice, as users may not use the applications as the developer expects. Even if this is possible,
current approaches all require to evaluate all candidate compiler options on the same input. Doing so across a range of inputs would incur
significant overhead, which thus prevents iterative compilation to be adopted at a larger scale.

Recent studies~\cite{chen12b,fang15} have used user inputs to perform {\itercomp} in distributed data centers. Using these approaches, each
compilation worker receives a subset of the input dataset to evaluate a small set of optimization settings on. The best such setting from
each round is used for subsequent executions of the same code. The best-found compilation strategy is then refined over time, by testing
new settings and re-evaluating old ones on new datasets. Like in~\cite{chen10} every setting is tested on the same set of inputs, wasting
time and energy. This approach only works well with MapReduce-like workloads, since it relies on the framework for repeating the same
computation multiple times without causing side-effects.

Older works attempted to drive \itercomp with execution time information~\cite{fursin:hal-01257279,fursin:inria-00001054}. They create
multiple versions of the hottest procedures, optimized and unoptimized, and at runtime randomly and dynamically select which one to use.
For each invocation, they record the time spent in the procedure and the version used. By comparing the average time required for each
invocation, they are able to evaluate the relative performance of different optimizations. The limitations of this approach are that it
depends on the amount of work performed by the procedure not changing dramatically from invocation to invocation and the number of
invocations being sufficiently high to collect multiple data points for each version. These drawbacks keep it from being a general solution
to the problem of iterative compilation.

Fursin~\etal~\cite{fursin07} attempted to find such a general approach. They proposed to use instructions per cycle (IPC)
as a metric for comparing different runs of the same program with different optimization settings and different inputs. Their hypothesis is
that application binaries displaying higher IPC are more efficient than lower IPC binaries. However, many optimization techniques affect
IPC in the opposite way, making the program more efficient while lowering IPC. The authors in~\cite{fursin07} acknowledge that this limits
the applicability of IPC for \itercomp. Our experiments in Section~\ref{sec:motivation} also show that IPC is not an appropriate metric
for quantifying efficiency for iterative compilation. In the context of multithreaded applications, IPC cannot be used to compare the performance of
different runs even when they use the same intput~\cite{alameldeen06,eyerman08}. Work-related metrics are shown to be more appropriate, but prior work
relied on application-specific work metrics. This paper presents the first automatic and application agnostic approach for deriving a work metric.


%PP: Discuss input sensitivity?

%\subsection{Work and Input Size Metrics}

%Previous work have proposed profiling-based mechanism to estimate input sizes~\cite{zaparanuks12,coppa14}.
%Coppa~\etal~\cite{coppa14} in particular propose the concept of \textit{read memory size} for automatically estimating the size of the input passed to a routine, where \textit{read memory size} represents the number of distinct memory cells first accessed by a read operation.
%In other words, the \textit{read memory size} metric measures the size of the useful portion of the input's memory footprint.
%However, because we are interested in the amount of computational work performed in respect of a given input, the memory footprint of the input may not always have a direct correspondence to  the amount of computational work.

%Goldsmith~\etal~\cite{goldsmith07} use \textit{block frequency} as the measure for performance for empirically describing the asymptotic behavior of programs, which is known as empirical computational complexity.
%Block frequency is a relative metric that represents the number of times a basic block executes~\cite{ball94,ball96}.
%They argue in favor of block frequency due to its portability, repeatability and exactness, since it does not suffer from timer resolution problems or non-deterministic noises.
%Block frequency also has the advantage of being efficiently profiled by means of automatic code instrumentation~\cite{knuth73,ball94}.

%However, in the context of comparing different optimizations, although block frequency would be able to capture aspects of optimizations that simplify the control-flow graph (CFG), measuring work at the basic block resolution would not capture effects of optimizations at the instruction level.
%Because of that, we extend the idea of using basic block frequency to measure computational work by also considering the computational cost of each basic block.
%The computational cost of a basic block is given by weighing the instructions that it contains.

\subsection{Optimal Instrumentation}

The seminal work of Donald Knuth~\cite{knuth73} introduced an optimal algorithm for profiling block frequency, inserting much fewer probes
than the na\"ive instrumentation approaches that were used in practice~\cite{knuth71}. Forman~\cite{forman81} proposed an approach to
further reduce the instrumentation overhead by placing probes in basic blocks that are less likely to be executed based on static
heuristics. Ball and Larus~\cite{ball94} offered a detailed discussion to compare these two approaches for optimally profiling block
frequency, namely, placing probes on the vertices or the edges of the control-flow graph (CFG). They show that the edge-based approach
produces better placement of probes. Our work built upon these prior works. We use the edge-based approach to place our probes before
applying relaxation.
